\documentclass{article}
\usepackage{amsmath}
\usepackage{url}
\usepackage{soul}
\usepackage[bookmarks,bookmarksopen,bookmarksdepth=2,colorlinks=false, pdfborder={0 0 0}]{hyperref}

\RequirePackage[colorinlistoftodos]{todonotes}

\let\quoteOld\quote
\let\endquoteOld\endquote
\renewenvironment{quote}{\quoteOld\itshape}{\endquoteOld}

\newcommand{\AS}[1]{\todo{[AS]: #1}}
\newcommand{\BG}[1]{\todo{[BG]: #1}}


\newcommand{\ASi}[1]{\todo[inline]{[AS]: #1}}
\newcommand{\BGi}[1]{\todo[inline]{[BG]: #1}}

\newcommand{\hlbg}[1]{{\sethlcolor{lightgray}\hl{#1}}}
\newcommand{\hlas}[1]{{\sethlcolor{lightblue}\hl{#1}}}

% use the following command line to convert to standard text file
% pandoc SPECIFICATION.tex -o SPECIFICATION.txt
%\setlength{\parindent}{0pt}

\title{ORE 2014 Specification}

\begin{document}


\maketitle

This document describes the reasoner specification (primarily input and output formats) for the ORE 2014 Competition (see \url{http://vsl2014.at/pages/ORE-index.html}).


\section{Reasoning Tasks}

For ORE 2014, the following reasoning tasks are evaluated:
\begin{itemize}
\item Classification: arranging all classes of an ontology in a class hierarchy according to their subsumption relations.
\item Consistency: checking the consistency of an ontology.
\item Realisation: identifying the types of all individuals in an ontology.
\end{itemize}
The reasoning task are evaluated for those OWL 2 Profiles for which enough ontologies can be gathered and enough profile-specific reasoners participate in the ORE 2014 Competition.

\section{Execution Environment}

The evaluation will be carried out on a Linux 64bit system. 
For this, it is required that the reasoning systems must fulfil the following specifications:
\begin{itemize}
\item The reasoning systems must be executable on older Linux 64bit systems (e.g., Fedora 12, Java version 1.6).
\item The reasoners must be encapsulated by a (shell) wrapper script, i.e., the execution of the wrapper script must trigger the reasoner.
\item The reasoners should not use a hard-coded memory/time limit.
\end{itemize}


\section{Input Format}

The input format for the ORE 2014 Competition is specified as follows:
\begin{itemize}
\item The encoding format for ontologies and command line arguments is \mbox{UTF-8}.
\item The ontologies are available in all fully fledged OWL 2 serialisation formats, i.e., reasoners can use the OWL 2 Functional Style, the OWL 2 XML, or the RDF/XML serialisation format.
\item Ontologies may contain datatypes, but do not contain DL-safe/SWRL rules.
\item All file paths are absolute, contain only ASCII-characters that are valid for files, and do not contain blank characters.
\item The wrapper script is executed with the following arguments:
\begin{enumerate}
\item The name of the reasoning task (\textless Operation\textgreater), i.e., either 'classification', 'consistency', or 'realisation'.
\item Input file path of the ontology (\textless OntologyFile\textgreater).
\item Output file path for the result (\textless Output\textgreater).
\end{enumerate}
For example, a wrapper script 'execReasoner.sh' is started with the command line: \\
./execReasoner.sh classification /ore/ont/pizza.owl /ore/out/result.dat
\end{itemize}




\section{Output Format}

The output format for the ORE 2014 Competition is specified as follows:
\begin{itemize}
\item The results must be encoded with \mbox{UTF-8}.
\item Warnings/errors thrown by the reasoner must be written into the file \mbox{\textless Output\textgreater\_err}, i.e., the given output file suffixed with '\_err'. In particular, if the processing of the given ontology might be incomplete (e.g., the ontology contains axioms that are not supported by the reasoner), then this should be reflected by corresponding error/warning messages.
\item It is guaranteed that all directories for the output files exist. If a file already exists, then the reasoner should overwrite the file.
\item The reasoner should report to the console output (to stdout):
\begin{enumerate}
\item The start message: 'Started \textless Operation\textgreater{} on \textless OntologyFile\textgreater'.
\item The operation time: 'Operation time: \textless Time\textgreater'.
\item The completion message: 'Completed \textless Operation\textgreater{} on \mbox{\textless OntologyFile\textgreater'}.
\end{enumerate}
\item The operation time (\textless Time\textgreater) should be measured in wall clock time, and should only be the value in milliseconds. 
Moreover, the operation time should represent the time elapsed from the moment preceding reasoner creation to the completion of the task at hand. 
That is, do not include ontology parsing time (unless some reasoner-specific pre-processing is done at this point), nor file serialization or socket communication time (where applicable).
\item The result must be written to the \textless Output\textgreater{} file and should be as defined below for the different reasoning tasks.
\end{itemize}

\subsection{Consistency Result Output Format}
For consistency, the result output file (\textless Output\textgreater{}) must contain 'true' if the ontology is consistent and 'false' if the ontology is inconsistent.

\subsection{Classification Result Output Format}
For classification, the result output file (\textless Output\textgreater{}) must be an OWL 2 file (in OWL 2 Functional Syntax, or any other serialization format that can be read with the OWL API), which contains the class hierarchy with EquivalentClasses and SubClassOf axioms, i.e.,
\begin{itemize}
\item Classes that are computed as semantically equivalent, must be expressed as equivalent by EquivalentClasses/SubClassOf axioms.
\begin{itemize}
\item For example, if A and B are equivalent, then this equivalence can be expressed by EquivalentClasses(A B) or also by the axioms SubClassOf(A B) and SubClassOf(B A).
\end{itemize}
\item Classes that are direct sub-classes of other classes must be expressed as sub-classes of these other classes by SubClassOf axioms.
\begin{itemize}
\item For instance, if A is a direct sub-class of B (i.e., there is no class C such that C is a super-class of A and a sub-class of B, and C is not equivalent to A or B), then this sub-class relationship can be expressed by SubClassOf(A B).
\end{itemize}
\item SubClassOf(owl:Nothing A) axioms can be omitted.
\item Note, if a class A is a direct sub-class of owl:Thing, then the result ontology must also contain SubClassOf(A owl:Thing).
\item If the ontology is inconsistent, then owl:Thing is equivalent to owl:Nothing, i.e., the result ontology should contain SubClassOf(owl:Thing owl:Nothing) or EquivalentClasses(owl:Thing owl:Nothing).
\item Note that for inconsistent ontologies the sub-class relationships of all other classes can be omitted.
\item Indirect sub-class relationships MUST NOT be expressed, i.e., the reasoner has to perform the transitive reduction.
\end{itemize}

\subsection{Realisation Result Output Format}
For realisation, the result output file (\textless Output\textgreater{}) must be an OWL 2 file (in OWL 2 Functional Syntax, or any other serialization format that can be read with the OWL API), which contains all types of all individuals with ClassAssertion axioms, i.e.,
\begin{itemize}
\item If an individual a is an instance of a class A, then the result ontology has to contain the axiom ClassAssertion(A a), i.e., all indirect types of individuals must also be expressed by ClassAssertion axioms.
\item If the ontology is inconsistent, then the result ontology must contain an axiom of the form SubClassOf(owl:Thing owl:Nothing), EquivalentClasses(owl:Thing owl:Nothing), or ClassAssertion(owl:Nothing a).
\end{itemize}



\end{document}

